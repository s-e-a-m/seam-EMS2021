%!TEX TS-program = xelatex
%!TEX encoding = UTF-8 Unicode
%!TEX root = 2020-GS-ARTICLE.tex
%----------------------------------------------------------------- LANGUAGES ---
\newcommand{\mylanguages}{italian,english} % in reverse order
%---------------------------------------------------------- TITLE & SUBTITLE ---
\newcommand{\mytitle}{SEAM PROJECT}
\newcommand{\mysubtitle}{Sustained ElectroAcoustic Music}
%----------------------------------------------------------------- AUTHOR(s) ---
\newcommand{\authorone}{Giovanni Michelangelo D'Urso}
\newcommand{\institutione}{Conservatorio S. Cecilia of Roma}
\newcommand{\emailone}{giovannimick94 @ gmail.com}
\newcommand{\phoneone}{}
%-------------------------------------------------------------------------------
\newcommand{\authortwo}{Giuseppe Silvi}
\newcommand{\institutiontwo}{Conservatorio N. Piccinni of Bari}
\newcommand{\emailtwo}{silvi.giuseppe @ docenticonsba.it} % duplicate these 3 lines if more
\newcommand{\phonetwo}{}
%-------------------------------------------------------------------------------
\newcommand{\authorthree}{Davide Tedesco}
\newcommand{\institutionthree}{Conservatorio S. Cecilia of Roma}
\newcommand{\emailthree}{me @ davidetedesco.it} % duplicate these 3 lines if more
\newcommand{\phonethree}{}
%-------------------------------------------------------------- STYLE GS2020 ---
% \newcommand{\authortwo}{Wikio Orgopedio}
% \newcommand{\institutiontwo}{Conservatorio S. Cecilia di Roma}
% \newcommand{\emailtwo}{wikio @ orgopedio.com} % duplicate these 3 lines if more
%-------------------------------------------------------------- STYLE GS2020 ---
\input{gs2020.tex}
%------------------------------------------------------------ BEGIN DOCUMENT ---
\begin{document}
\maketitle
\thispagestyle{empty}

Sustained ElectroAcoustic Music
\footnote{\url{http://s-e-a-m.github.io}}\footnote{\url{http://seam-world.slack.com}}
is an open-source project inspired by Alvise Vidolin and Nicola Bernardini's
article \cite{bevi05} on live electroacoustic music sustainability. Their text
points at the sustainability problem's multiple technical faces such as
technological, notational or general conception issues. Even if the article
aforementioned focuses only on live electroacoustic music, the concept of
sustainability applies to any documented music that uses electroacoustic
environments, including the acousmatic works, instruments mixed with tape, and
structured amplified work the purpose of the presented text. This project aims
to grow the interpretation and electroacoustic musical practice with the
consciousness of electronic and informatics issues.

Almost one hundred years ago, Ottorino Respighi introduced a recorded media into
his orchestral composition \emph{I Pini di Roma} \cite{ropr25} and, even today,
we do not have a shared consolidate electroacoustic practice to play it likewise
the orchestral one. The problem is more profound if we consider that most of
today's electroacoustic manipulators do not know who Respighi was and the
differences between his pioneer usage of recordings, instead the later
compositional purpose usage made by John Cage. \cite{cjil39}

Electronics and informatics introduction in composition changed the music industry inexorably and transformed the playing and production approach. We are not
speaking about the inevitable technologic half of those facts, but of the
musical one, built on literature and interpretation.

Sustainable musical activity stems from the ambition to perform and interpret an electroacoustic work without rebuilding the instruments every time. This
activity has led to the exclusive treatment of technical aspects, with the
consequent separation of the electroacoustic medium's significance from the
composition's poetic and executive aspects. The SEAM community's purpose is to
collect musical instructions, establish them, and share them for more agile
musical interpretations often hampered by the same technical aspects, ensuring
the possible stratification of information following multiple performances.

The process of making a piece sustainable is to collect and pass on the helpful information for performance at the highest possible level, stimulating and
educating the composer himself. In this way, it will be possible to create
electronics that are technically independent and infinitely reproducible.

\vfill\null

\raggedright
\bibliographystyle{unsrt}
\bibliography{includes/bibliography.bib}

% \clearpage
%
% \includepdf[pages=-]{includes/cvs.pdf}

\end{document}

%%%%%%%%%%%%%%%%%%%%%%%%%%%%%%%%%%%%%%%%%%%%%%%%%%%%%%%%%%%%%%%%%%%%%%%%%%%%%%%%
% 2020 GIUSEPPE SILVI ARTICLE TEMPLATE BASED ON
%%%%%%%%%%%%%%%%%%%%%%%%%%%%%%%%%%%%%%%%%%%%%%%%%%%%%%%%%%%%%%%%%%%%%%%%%%%%%%%%
% Journal Article
% LaTeX Template
% Version 1.4 (15/5/16)
% This template has been downloaded from:
% http://www.LaTeXTemplates.com
% Original author:
% Frits Wenneker (http://www.howtotex.com) with extensive modifications by
% Vel (vel@LaTeXTemplates.com)
% License:
% CC BY-NC-SA 3.0 (http://creativecommons.org/licenses/by-nc-sa/3.0/)
%%%%%%%%%%%%%%%%%%%%%%%%%%%%%%%%%%%%%%%%%%%%%%%%%%%%%%%%%%%%%%%%%%%%%%%%%%%%%%%%
