%!TEX TS-program = xelatex
%!TEX encoding = UTF-8 Unicode
%!TEX root = 2020-GS-ARTICLE.tex
%----------------------------------------------------------------- LANGUAGES ---
\newcommand{\mylanguages}{italian,english} % in reverse order
%---------------------------------------------------------- TITLE & SUBTITLE ---
\newcommand{\mytitle}{SEAM PROJECT}
\newcommand{\mysubtitle}{Sustained ElectroAcoustic Music}
%----------------------------------------------------------------- AUTHOR(s) ---
\newcommand{\authorone}{Giovanni Michelangelo D'Urso}
\newcommand{\institutione}{Conservatorio S. Cecilia of Roma}
\newcommand{\emailone}{giovannimick94 @ gmail.com}
\newcommand{\phoneone}{}
%-------------------------------------------------------------------------------
\newcommand{\authortwo}{Giuseppe Silvi}
\newcommand{\institutiontwo}{Conservatorio N. Piccinni of Bari}
\newcommand{\emailtwo}{silvi.giuseppe @ docenticonsba.it} % duplicate these 3 lines if more
\newcommand{\phonetwo}{}
%-------------------------------------------------------------------------------
\newcommand{\authorthree}{Davide Tedesco}
\newcommand{\institutionthree}{Conservatorio S. Cecilia of Roma}
\newcommand{\emailthree}{me @ davidetedesco.it} % duplicate these 3 lines if more
\newcommand{\phonethree}{}
%-------------------------------------------------------------- STYLE GS2020 ---
% \newcommand{\authortwo}{Wikio Orgopedio}
% \newcommand{\institutiontwo}{Conservatorio S. Cecilia di Roma}
% \newcommand{\emailtwo}{wikio @ orgopedio.com} % duplicate these 3 lines if more
%-------------------------------------------------------------- STYLE GS2020 ---
%!TEX TS-program = xelatex
%!TEX encoding = UTF-8 Unicode
%!TEX root = 2020-GS-ARTICLE.tex
%-------------------------------- PACKAGES AND OTHER DOCUMENT CONFIGURATIONS ---
\documentclass[
	a4paper,
	twocolumn
]{article}
\usepackage[
	top=20mm,
	bottom=25mm,
	textwidth=17.2cm,
	columnsep=0.8cm
]{geometry}
\usepackage[T1]{fontenc}
\usepackage[\mylanguages]{babel}
\usepackage{graphicx}
\usepackage{dblfloatfix}
\usepackage{graphicx}
\usepackage{epstopdf}
\epstopdfsetup{update}
\usepackage[usenames]{color}
\usepackage{xcolor}
\usepackage{amssymb}
\usepackage{hyperref} % For hyperlinks in the PDF
\usepackage{pdfpages}
\usepackage{Alegreya}
\linespread{1.05}
\usepackage{
	fontspec,
	xltxtra,
	xunicode
	}
\usepackage{
	xfrac,
	unicode-math
	}

\defaultfontfeatures{Mapping=tex-text}
\setmonofont[
	Scale=MatchLowercase
	]{Andale Mono}
\setmathfont[
	Scale=MatchLowercase,
	Scale=1
	]{Libertinus Math}

\usepackage{microtype}

\usepackage[
	hang,
	small,
	labelfont=bf,
	up,
	textfont=it,
	up
	]{caption}
\usepackage{paralist} % For compact item lists
\usepackage{etoolbox} % Some tools: used for quote environment
\AtBeginEnvironment{quote}{\small}
\usepackage{titling} % Customizing the title section
\usepackage{booktabs} % Horizontal rules in tables
\usepackage{enumitem} % Customized lists
\setlist[itemize]{noitemsep} % Make itemize lists more compact
\usepackage{abstract} % Allows abstract customization
\renewcommand{\abstractnamefont}{\normalfont\bfseries} % Set the "Abstract" text to bold
\renewcommand{\abstracttextfont}{\normalfont\small\itshape} % Set the abstract itself to small italic text
\usepackage{titlesec} % Allows customization of titles
\renewcommand\thesection{\Roman{section}} % Roman numerals for the sections
\renewcommand\thesubsection{\Roman{subsection}} % roman numerals for subsections
\titleformat{\section}[block]{\large\centering}{\thesection.}{1em}{} % Change the look of the section titles
\titleformat{\subsection}[block]{\large}{\thesubsection.}{1em}{} % Change the look of the section titles
%------------------------------------------------------------- TITLE SECTION ---
\setlength{\droptitle}{-4\baselineskip} % Move the title up
\pretitle{\begin{center}\huge\bfseries} % Article title formatting
\posttitle{\end{center}} % Article title closing formatting
\title{\mytitle \\[0.1cm] \large{\emph{\mysubtitle}}} % Article title
\author{%
\textsc{\authorone}\\%
\normalsize \institutione \\ %
\normalsize \emailone \\ %
\normalsize \phoneone %
% activate
\and % duplicate these 4 lines if more
\textsc{\authortwo} \\%
\normalsize \institutiontwo \\ %
\normalsize \emailtwo \\%
\normalsize \phonetwo %
\and % duplicate these 4 lines if more
\textsc{\authorthree} \\%
\normalsize \institutionthree \\ %
\normalsize \emailthree \\%
\normalsize \phonethree %
}
\date{} % Leave empty to omit a date

\usepackage{fancyhdr} % Headers and footers
\pagestyle{fancy} % All pages have headers and footers
\fancyhead{} % Blank out the default header
\fancyfoot{} % Blank out the default footer
\fancyhead[C]{\small Wikipedia • General Relativity} % Custom header text
\fancyfoot[RO,LE]{\small \today~ • w: \input{includes/words.txt} • c: \input{includes/char.txt} • p:~\thepage} % Custom footer text
%-------------------------------------------------------------------------------
%-------------------------------------------------------------------------------
%	LISTINGS
%-------------------------------------------------------------------------------
%-------------------------------------------------------------------------------
\usepackage{listings}
% lstlistings setup
\definecolor{gsbg}{rgb}{0.98,0.98,0.98}

\lstset{%
  aboveskip=10pt,
	belowskip=5pt,
  language=C++,
  numbers=none,%left,%none,
  tabsize=4,
  %frame=single,
  breaklines=true,
  numberstyle=\tiny\ttfamily,
  backgroundcolor=\color{gsbg},
  basicstyle=\footnotesize\ttfamily,
  %commentstyle=\slshape\color{mylstcmt}, %\itshape,
  %frameround=tttt,
  columns=flexible, %fixed,
  showstringspaces=false,
  emptylines=2,
  inputencoding=utf8,
  extendedchars=true,
  literate=	{á}{{\'a}}1
			{à}{{\`a}}1
			{ä}{{\"a}}1
			{â}{{\^a}}1
			{é}{{\'e}}1
			{è}{{\`e}}1
			{ë}{{\"e}}1
			{ê}{{\^e}}1
			{ï}{{\"i}}1
			{î}{{\^i}}1
			{ö}{{\"o}}1
			{ô}{{\^o}}1
			{è}{{\`e}}1
			{ù}{{\`u}}1
			{û}{{\^u}}1
			{ç}{{\c{c}}}1
			{Ç}{{\c{C}}}1,
  emph={component, declare, environment, import, library, process},
  emph={[2]ffunction, fconstant, fvariable},
  emph={[3]button, checkbox, vslider, hslider, nentry, vgroup, hgroup, tgroup, vbargraph, hbargraph, attach},
  %emphstyle=\color{yotxt}, %\underline, %\bfseries,
  %morecomment=[s][\color{mylstdoc}]{<mdoc>}{</mdoc>},
  rulecolor=\color{black}
}

\usepackage[framemethod=tikz]{mdframed} % Allows defining custom boxed/framed environments

%-------------------------------------------------------------------------------
%--------------------------------------------------- INFORMATION ENVIRONMENT ---
%-------------------------------------------------------------------------------

% Usage:
% \begin{info}[optional title, defaults to "Info:"]
% 	contents
% 	\end{info}

\mdfdefinestyle{info}{%
	topline=false, bottomline=false,
	leftline=false, rightline=false,
	nobreak,
	singleextra={%
		\fill[black](P-|O)circle[radius=0.4em];
		\node at(P-|O){\color{white}\scriptsize\bf i};
		\draw[very thick](P-|O)++(0,-0.8em)--(O);%--(O-|P);
	}
}

% Define a custom environment for information
\newenvironment{info}[1][Info:]{ % Set the default title to "Info:"
	\medskip
	\begin{mdframed}[style=info]
		\noindent{\textbf{#1}}
}{
	\end{mdframed}
}

%-------------------------------------------------------------------------------
%----------------------------------------------------- BIOGRAFIA ENVIRONMENT ---
%-------------------------------------------------------------------------------

% Usage:
% \begin{bio}[optional title, defaults to "Info:"]
% 	contents
% 	\end{bio}

\mdfdefinestyle{bio}{%
	topline=false, bottomline=false,
	leftline=false, rightline=false,
	nobreak,
	singleextra={%
		\fill[black](P-|O)circle[radius=0.4em];
		\node at(P-|O){\color{white}\scriptsize\bf b};
		\draw[very thick](P-|O)++(0,-0.8em)--(O);%--(O-|P);
	}
}

% Define a custom environment for information
\newenvironment{bio}[1][Biografia:]{ % Set the default title to "Info:"
	\medskip
	\begin{mdframed}[style=bio]
		\noindent{\textbf{#1}}
}{
	\end{mdframed}
}

%-------------------------------------------------------------------------------
%------------------------------------------------------- WARNING ENVIRONMENT ---
%-------------------------------------------------------------------------------

% Usage:
% \begin{warn}[optional title, defaults to "Warning:"]
%	Contents
% \end{warn}

\mdfdefinestyle{warning}{
	topline=false, bottomline=false,
	leftline=false, rightline=false,
	nobreak,
	singleextra={%
		\draw(P-|O)++(-0.5em,0)node(tmp1){};
		\draw(P-|O)++(0.5em,0)node(tmp2){};
		\fill[black,rotate around={45:(P-|O)}](tmp1)rectangle(tmp2);
		\node at(P-|O){\color{white}\scriptsize\bf !};
		\draw[very thick](P-|O)++(0,-1em)--(O);%--(O-|P);
	}
}

% Define a custom environment for warning text
\newenvironment{warn}[1][Warning:]{ % Set the default warning to "Warning:"
	\medskip
	\begin{mdframed}[style=warning]
		\noindent{\textbf{#1}}
}{
	\end{mdframed}
}

%-------------------------------------------------------------------- ABSTRACT -
% \renewcommand{\maketitlehookd}{%
% \begin{abstract}
% \noindent\input{includes/abstract.txt}
% \end{abstract}
% }

%------------------------------------------------------------ BEGIN DOCUMENT ---
\begin{document}
\maketitle
\thispagestyle{empty}

Sustained ElectroAcoustic Music
\footnote{\url{http://s-e-a-m.github.io}}\footnote{\url{http://seam-world.slack.com}}
is an open-source project inspired by Alvise Vidolin and Nicola Bernardini's
article \cite{bevi05} on live electroacoustic music sustainability. Their text
points at the sustainability problem's multiple technical faces such as
technological, notational or general conception issues. Even if the article
aforementioned focuses only on live electroacoustic music, the concept of
sustainability applies to any documented music that uses electroacoustic
environments, including the acousmatic works, instruments mixed with tape, and
structured amplified work the purpose of the presented text. This project aims
to grow the interpretation and electroacoustic musical practice with the
consciousness of electronic and informatics issues.

Almost one hundred years ago, Ottorino Respighi introduced a recorded media into
his orchestral composition \emph{I Pini di Roma} \cite{ropr25} and, even today,
we do not have a shared consolidate electroacoustic practice to play it likewise
the orchestral one. The problem is more profound if we consider that most of
today's electroacoustic manipulators do not know who Respighi was and the
differences between his pioneer usage of recordings, instead the later
compositional purpose usage made by John Cage. \cite{cjil39}

Electronics and informatics introduction in composition changed the music industry inexorably and transformed the playing and production approach. We are not
speaking about the inevitable technologic half of those facts, but of the
musical one, built on literature and interpretation.

Sustainable musical activity stems from the ambition to perform and interpret an electroacoustic work without rebuilding the instruments every time. This
activity has led to the exclusive treatment of technical aspects, with the
consequent separation of the electroacoustic medium's significance from the
composition's poetic and executive aspects. The SEAM community's purpose is to
collect musical instructions, establish them, and share them for more agile
musical interpretations often hampered by the same technical aspects, ensuring
the possible stratification of information following multiple performances.

The process of making a piece sustainable is to collect and pass on the helpful information for performance at the highest possible level, stimulating and
educating the composer himself. In this way, it will be possible to create
electronics that are technically independent and infinitely reproducible.

\section*{CASE STUDIES}

% Developing the instrument and instrumentalist concepts to the combined form of those into interpretation requires overcoming obsolete parallelism: the computer music performer as an artisan of new luthiery. There is not a sustainability conception under the deception of that wrong and obsolete metaphor. Each luthiery is new, and it evolves with musical needs. Each instrument has his inventor and his virtuoso, but in musical history, those people never coincided. The best instruments were conceived from men entirely devoted to the conception of something unique. The best virtuoso took those instruments to unveil their perspective.

Referring to the electroacoustic music literature, where the substantial difference with the acoustical one is an inevitable continuously changing environment, we prefer to use the topology classification. According to general type, a typology classification is used where characteristics of something are fixed and produce a catalogue of things. A topology classification considers the time-space characteristics of shape instead and permits the time variance of the environments. We classify three topologies of electroacoustic music in literature:

\begin{description}
  \item[The undocumented] where composers use only word description to generate environment and circumstances;
  \item[The \emph{hole-word}]\footnote{in stud poker a card which has been dealt
    face down.} where the score has deep technical documentation but listing names of undocumented instruments;
  \item[The coded] where informatics translations between languages or informatics technologies are based on literature and shared knowledge.
\end{description}

The identification of topological classes in place of typological forms is
necessary to subordinate technology-matter to the musical practice and poetics.

\subsection*{Write the undocumented}

\subsubsection*{1969, \emph{I am Sitting in a Room}, Alvin Lucier}

Speaking at beginner music students about Lucier's \emph{I am sitting in a room}, is a kind of sharing of a multilevel experience. There are many access layers, each with different bits of knowledge requirements. One of these, of course, is how it could be done today.

The score states a text to be read; it explains what will happen and why, so it unveils the process itself. The acoustical properties of the space transform the speech. The \emph{resonant frequency of the room reinforces themselves}, while the others are absorbed, they are attenuated by space, as an instrument to be played and articulated by time.

Today the work could be realised with live electronics, without interruption between cycles. It requires a simple delay line, sized as much as the statement, to be infinitely recycled. Again, today, the sensibility that had characterised the \emph{Lucier's Era} is overwhelmed by anxiety and incapacity to observe something in time. The time-lapse perception model is a \emph{state of mind} constriction, so, with the maximum technological support of an infinite digital delay line, without the necessity of an entire perception, \emph{I am sitting} could be a surgical time-waste, at best quality, of course. The idea of space as an instrument expressed by this apostolic work requires ears and fingers twisted in a whole participated perception of time-space mutation during the performance.

The severe sustainability problem of that work is not technical. It is a simple process. The very most profound problem is sensibility. The worst thing that can happen to the \emph{process-music} is the perfect process execution without the music. To seam process and music, we need to unfold ears and minds to Lucier's perception and sensibility. How? Doing it, like he exactly suggested fifty years ago: practising.

\begin{quote}
Make versions in which one recorded statement is recycled through many rooms.
Make versions using one or more speakers of different languages in different
rooms. Make versions in which, for each generation, the microphone is moved to
different parts of the room or rooms. Make versions that can be performed in
real-time. \cite{lais69}
\end{quote}

% The many versions proposed by the author in the music score point to multiple cases of electroacoustic staging. It is a \emph{free-your-electroacoustic-fantasy} statement, typical of the end of the sixties, unfortunately, forgot the day after. To reset future people's perception by now, young musicians should do that for years, never mapping one single patch in \emph{max4live}. They need an instrument to practise music.

% The offline process remains, like the original statement says, really unchanged. A double recording apparatus of any nature and a chair to sit down and practise.

\subsection*{Rewrite}

\subsubsection*{1987, \emph{Post-prae-ludium n. 1 per Donau}, Luigi Nono}
\subsubsection*{1989, \emph{Risonanze Erranti}, Luigi Nono}

To avoid misunderstanding, every technological rewriting based on a block diagram must be partially true. Each block named with an intergalactic hole-word can lead everywhere. An example: the sound of the \emph{Halaphon} (to cite one of the Nono's hole-word block) not exist. The \emph{Halaphon} was a way to connect pure musical thinking with consolidated musical practice, embracing acoustical space and electronics. Before the instrument itself, the \emph{Halaphon} was an idea of space-related music that became a necessity and only at the end a technological opportunity.

% \begin{quote}
% The Halaphon is a digital spatializer which, with the loudspeakers arranged in the
% room, controls the movement of sound in space. This movement must be continuous,
% with a soft, superimposed fading from one loudspeaker to another. The dynamics
% indicated in the score for the Contralto and the instruments also apply for the
% dynamics of the Halaphon outputs \cite{nlre87}. %Overall four interventions with
% %the Halaphon are foreseen (see Diagrams 11, 13, 16 and 19): with the exception
% %of the intervention of programme 16, they are all closely linked to the
% %rhythmic-temporal execution of the voice or instruments. Further information on
% %programming the Halaphon is to be found further on, in the paragraph Special
% %Information, in the diagrams and notations in the score.
% \end{quote}

% Words can define and reduce some issues to integrate diagrams and musical notation where they are inconsistent or obscure. Reading a musical score unknowing the mental state of the composer that brought it to the world, is a daily committed crime \cite{rw05}. A composer poetics is inevitably invoked into the work he is producing and his musical practice and research. Even when there are knowledge and structured
% thinking, even then, we can produce the wrong questions to obtain the right, not necessary, answer.

If there is something that must be sustained is exactly that musical behaviour. Each of Nono's words conducts the musician to agile and deeply performable electroacoustic musical environments. So it passes the concept of an instrumental practice consolidated on the means and tools available. Nono himself talks about it by transversally crossing architecture, classical musical practice and technology, in executive and interpretative terms.

Musical score thinking and annotating procedure to obtain a musical match, like Nono's procedure of studying, practising, listening and writing once between infinite possibilities, declare itself as an attempt to do something not complete or not fully defined in its process.

\subsection*{Porting}

The porting of aged music informatics of experimental instruments to a sustained programming language and technology merging into two branches of interests of the authors: the history of instruments (even the technological ones) and the revival of music lost in the past for technological issues, into a new possibility of music playing.

\subsubsection*{1991, \emph{Mobile Locale}, Michelangelo Lupone}

Working side by side with Michelangelo Lupone for the \emph{Mobile Locale}
\cite{lmml91} porting is exceptionally musical and only marginally a technological and informatics matter. The main goal is to interpret his music with his unavoidable sensibility at the disposal of better comprehension of the music score. Close to this, the fascinating possibility to revive a beautiful work, \emph{Mobile Locale}, stuck by technical problems that obscured its musical value. The work is for percussion and live electronics with fixed media and was conceived around a technology born at CRM\footnote{\url{http://www.crm-music.it/}}
(Centro Ricerche Musicali, in Rome), the \emph{System Fly} \cite{ml85}.

\vfill\null

\raggedright
\bibliographystyle{unsrt}
\bibliography{includes/bibliography.bib}

% \clearpage
%
% \includepdf[pages=-]{includes/cvs.pdf}

\end{document}

%%%%%%%%%%%%%%%%%%%%%%%%%%%%%%%%%%%%%%%%%%%%%%%%%%%%%%%%%%%%%%%%%%%%%%%%%%%%%%%%
% 2020 GIUSEPPE SILVI ARTICLE TEMPLATE BASED ON
%%%%%%%%%%%%%%%%%%%%%%%%%%%%%%%%%%%%%%%%%%%%%%%%%%%%%%%%%%%%%%%%%%%%%%%%%%%%%%%%
% Journal Article
% LaTeX Template
% Version 1.4 (15/5/16)
% This template has been downloaded from:
% http://www.LaTeXTemplates.com
% Original author:
% Frits Wenneker (http://www.howtotex.com) with extensive modifications by
% Vel (vel@LaTeXTemplates.com)
% License:
% CC BY-NC-SA 3.0 (http://creativecommons.org/licenses/by-nc-sa/3.0/)
%%%%%%%%%%%%%%%%%%%%%%%%%%%%%%%%%%%%%%%%%%%%%%%%%%%%%%%%%%%%%%%%%%%%%%%%%%%%%%%%
